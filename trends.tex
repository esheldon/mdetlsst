\begin{figure}
    \centering
\begin{tikzpicture}
    \begin{groupplot}[
        group style={
            group size=1 by 2,
            % vertical sep=1cm,
        },
        % xmin=0.0,
        % xmax=110.0,
        width=\columnwidth,
        height=0.7\columnwidth,
        minor tick num=3,
        axis on top,
        legend style={
            legend columns=2,
            draw=none,
        },
        % legend columns=2,
        % legend draw=none,
        % my fill legend
    ]

    %
    % bias range as function of S/N
    %
    \nextgroupplot[
        xlabel={Minimum S/N},
        ylabel={$m/10^{-3}$},
        xmin=9,xmax=21,
        % ymin=-3.5,ymax=3.5,
        ymin=-2.5,ymax=3.4,
        % xticklabels={,,},
    ]

    % mark level of nonlinear shear
	\addplot+[dashdotted,sand5,no markers,thick] coordinates {
        (9.0,0.4)
        (21.0,0.4)
    };
    \addlegendentry{Nonlinear Shear}


    % mark requirements
	\addplot+[name path=A,bisque,no markers,forget plot] coordinates {
        (9.0,-2)
        (21.0,-2)
    };
    \addplot+[name path=B,bisque,no markers,forget plot] coordinates {
        (9.0,2)
        (21.0,2)
    };

    % \addplot+[gray!30,opacity=0.5] fill between[of=A and B];
    \addplot+[bisque,opacity=0.5] fill between[of=A and B];
    \addlegendentry{Requirement}

    % bias range as function of S/N
	\addplot+[name path=mlowvar,desert3,no markers,solid,opacity=0.5,forget plot] coordinates {
        (10.0,0.219)
        (12.5,0.078)
        (15.0,0.213)
        (20.0,0.257)
    };
    \addplot+[name path=mhighvar,desert3,no markers,solid,opacity=0.5,forget plot] coordinates {
        (10.0,1.215)
        (12.5,0.869)
        (15.0,0.973)
        (20.0,0.892)
    };
    % black here turns out to be the boundary for the legend, a bug of some kind
    % \addplot+[color=black,pattern=north east lines,solid] fill between[of=mlowvar and mhighvar];
    \addplot+[color=desert3,opacity=0.5] fill between[of=mlowvar and mhighvar];
    % \addplot+[color=gray!50,opacity=0.5] fill between[of=mlowvar and mhighvar];
    \addlegendentry{LSST 10 year PSF}

	\addplot+[name path=mlowfix,bsand4,no markers,solid,opacity=0.5,forget plot] coordinates {
        (10.0,0.176295)
        (12.5,-0.00826672)
        (15.0,0.165007)
        (20.0,0.111993)
    };
    \addplot+[name path=mhighfix,bsand4,no markers,solid,opacity=0.5,forget plot] coordinates {
        (10.0,1.13744)
        (12.5,0.789173)
        (15.0,0.915298)
        (20.0,0.765478)
    };
    % \addplot+[color=black,pattern=vertical lines,solid] fill between[of=mlowfix and mhighfix];
    \addplot+[color=bsand4,opacity=0.5] fill between[of=mlowfix and mhighfix];
    % \addplot+[color=gray!75,opacity=0.5] fill between[of=mlowfix and mhighfix];
    \addlegendentry{Constant PSF}

	\addplot+[solid,black!80,no markers,forget plot] coordinates {
        (9.0,0.0)
        (21.0,0.0)
    };

    %
    % bias range as function of T/Tpsf
    %
    \nextgroupplot[
        xlabel={Minimum $T/T_{PSF}$},
        ylabel={$m/10^{-3}$},
        xmin=1.15,xmax=1.55,
        % ymin=-3.5,ymax=3.5,
        ymin=-2.5,ymax=3.4,
        % xticklabels={,,},
    ]

    % mark level of nonlinear shear
	\addplot+[dashdotted,sand5,no markers,thick] coordinates {
        (1.15,0.4)
        (1.55,0.4)
    };


    % mark requirements
	\addplot+[name path=A,bisque,no markers,forget plot] coordinates {
        (1.15,-2)
        (1.55,-2)
    };
    \addplot+[name path=B,bisque,no markers,forget plot] coordinates {
        (1.15,2)
        (1.55,2)
    };

    \addplot+[bisque,opacity=0.5] fill between[of=A and B];

    % bias range as function of T/Tpsf
	\addplot+[name path=mlowvar,desert3,no markers,solid,opacity=0.5,forget plot] coordinates {
        (1.2, 0.211992)
        (1.3, 0.151222)
        (1.4, -0.0532513)
        (1.5, -0.346328)

    };
    \addplot+[name path=mhighvar,desert3,no markers,solid,opacity=0.5,forget plot] coordinates {
        (1.2, 0.967327)
        (1.3, 1.04341)
        (1.4, 0.9606359999999999)
        (1.5, 0.968678)

    };
    % black here turns out to be the boundary for the legend, a bug of some kind
    % \addplot+[color=black,pattern=north east lines,solid] fill between[of=mlowvar and mhighvar];
    \addplot+[color=desert3,opacity=0.5] fill between[of=mlowvar and mhighvar];

    \addplot+[name path=Tratmlowfix,bsand4,no markers,solid,opacity=0.5,forget plot] coordinates {
        (1.2, 0.165007)
        (1.3, -0.0856819)
        (1.4, -0.114263)
        (1.5, -0.370727)
    };
    \addplot+[name path=Tratmhighfix,bsand4,no markers,solid,opacity=0.5,forget plot] coordinates {
        (1.2, 0.9152980000000001)
        (1.3, 0.799206)
        (1.4, 0.858479)
        (1.5, 0.9879929999999999)
    };
    % \addplot+[color=black,pattern=vertical lines,solid] fill between[of=mlowfix and mhighfix];
    \addplot+[color=bsand4,opacity=0.5] fill between[of=Tratmlowfix and Tratmhighfix];

	\addplot+[solid,black!80,no markers,forget plot] coordinates {
        (1.15,0.0)
        (1.55,0.0)
    };


    \end{groupplot}
\end{tikzpicture}

    \caption{
        Multiplicative shear bias $m$ for a simulation with a realistic galaxy
        sample, random galaxy layout, image artifacts and background
        subtraction errors.  The top panel shows the bias as a function of
        \snr\ and the bottom panel shows the bias as a function of the square
        size ratio \Tratio.  The shaded areas show the 99.7\%~confidence bands
        for the constant PSF (darker) and spatially variable PSF (lighter).
        The dot-dashed line represents the approximate expected bias due to
        higher order shear effects, and the broader shaded band represents our
        nominal accuracy goal.   In all cases the results are consistent with
        the expected bias.
    }
    \label{fig:trends}


\end{figure}


